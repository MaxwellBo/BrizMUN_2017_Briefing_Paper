\documentclass[11pt,a4paper]{report}

\usepackage{fullpage}
% \usepackage[margin=1in]{geometry}
\usepackage[utf8]{inputenc}

\usepackage{graphicx}
\graphicspath{ {images/} }

\usepackage[colorlinks = true,
            linkcolor = blue,
            urlcolor  = blue,
            citecolor = blue,
            anchorcolor = blue]{hyperref}

\begin{document}

\title{BrizMUN 2017 Briefing Paper \\ SPECPOL \\ Preventing Interstate Interference with Elections and Referendums}
\author{Maxwell Bo \& Dermot Aldenton}
\date{April 20, 2017}
\maketitle


\section{Background Information}

SPECPOL will investigate ways to prevent interstate interference in international elections and referendums. Corruption and fraud within domestic elections has been known to occur in developing countries, and they have been increasingly monitored by international bodies such as the United Nations. Interstate tampering and interference within other states’ domestic elections and referendums is an important issue especially within developing nations being influenced by larger more developed nations. SPECPOL will investigate current problems with election interference and the corruption and tampering of referendum results by interstate parties.

\subsection{The Committee}

The \textbf{Special Political and Decolonization Committee (SPECPOL)} deals with a variety of subjects which include those related to decolonisation, Palestinian refugees and human rights, peacekeeping, mine action, outer space, public information, atomic radiation and the University for Peace. 

General Assembly resolutions define important issues, reaffirm and condemn past actions, and recommend future action – in general, they do not have the jurisdiction to make binding decisions. Resolutions on this topic will require a simple majority (50\%+1) to pass.

\subsubsection{History}

\section{The Issue}
\subsection{History of the Issue}
The stable transition of power from one platform to another has throughout history been a controversial issue. Ensuring truthful and transparent power transitions and preserving the political rights of humans is a core ideal of the United Nations, one enshrined in the declaration of Human rights. As such, democratic election fraud, tampering and interference run contrary to the rights and ideals the United Nations stand for. Throughout the history of democratic election, election interference has played a role in corrupting the political power of nation-states and illegally claiming political rights and rule essentially, coup de tat's.

\subsubsection{Modern History}

International influence in democratic elections is a controversial issue, marred by subtle, ambiguity and general secrecy. However there are numerous elections in the 20th and 21st century purported to have been influenced in this manner; The Romanian 1946 Elections , The Ukraine 2010 presidential election, Crimean referendum and the most recent United States presidential elections. In recent years (21st century) more elections have become `controversial' due to fears of election tampering and due to poor anti-political corruption measures, lack of cohesive international agreement on election protections and understandings. 

\subsubsection{U.N. Approach to the Issue}

The United Nations recognizes through the Universal declaration of Human Rights and The International Covenant on Civil and Political Rights the unalienable right of every human to peaceful, transparent and `true' elections. Whilst recognizing the sovereign right each nation-state has to develop its own political systems, the U.N at the personal request of nation-states instituting elections to provide monitors and advice to the government and institution holding the elections through the Electoral Assistance Division (EAD).

\subsubsection{Existing Frameworks and Achievements}
The United Nations in resolutions past has addressed this tropic broadly, yet has failed to account for new technological innovation, special decolonized nation-states and instances where countries have violated past resolutions without consequence.  These will form the basis of a counter-response this committee will be addressing. 

\subsection{The Issues}

SPECPOL will investigate ways to prevent interstate interference in international elections and referendums.

\subsubsection{What constitutes interference?}

Clear and definitive definition needs to be agreed upon and ratified in order to set the scope of the investigation, pave the way for decisive action and create a standard to which all countries can measure their response too.

\subsubsection{Are there ways to supplement or improve existing measures to halt Interstate interference?}

Despite the many treaties urging, appealing or condemning actions of interstate interference, instances of interference have been on the rise in the 21st Century. As such, this committee must explore new avenues of approach to combat the practice and check its use.

\subsubsection{Can measures be created to increase protections of elections in developing countries?}
Developing countries in particular are at risk of election interference and fraud due to increased corruption and inadequate institution ability, experience or technology. Special care must be taken to ensure that these countries have policy that integrates into their fragile political environments

\subsubsection{What role will cyber-security play in new elections?}
With the beginning of electronic polling and electoral process's, increased scrutiny and measures will need to be drafted and formulated into a cohesive policy to combat the rise of electronic sabotage and interference primarily carried out by malicious computer code.

\subsection{Potential Solutions}

\begin{itemize}
	\item States should be aware of their past treaty obligations and responsibilities in maintaining international peace and security and preserving the beneficial status-quo for humanity.
	\item States should be focused on clearly developing a response for the breach of past treaties and resolutions.
	\item States should focus efforts to form a consensus on the concept of interference and a policy response to it.
	\item States must come to agreement on the paradigm of cyber-security in relation to this issue and a response to it.
	\item Individual states should be clear about their countries stances on each relevant issue bought up in chamber, these should form the basis for a cohesive comprehensive counter to the situation developed.
\end{itemize}


\subsection{Rules of Debate}

As a delegate, your primary goal is to be \textit{effective at achieving your countries goals}. Delegates should try to shape debate with working papers, amendments and negotation, to ensure that the resolution that is passed (or not passed) is satisfactory in the eyes of your country.

Ultimately, you should try to ensure that the goals you try to achieve in committee are the same goals your country has in reality. Good preparation and research is essential, so that your approach to tackling these goals is clear, nuanced and effective. Often, it will be difficult to find specific information on your countries perspectives on an issue - you'll likely need to extrapolate these perspectives, and provide adequate reasoning for these perspectives in debate. 

\section{Further Research}

In particular, delegates should have a general understanding of their nation's:

\begin{itemize}
	\item Existing policy on each subject area
	\item Political system and party in power
\end{itemize}

\end{document}
