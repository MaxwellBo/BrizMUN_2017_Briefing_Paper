\documentclass[11pt,a4paper]{report}

\usepackage{fullpage}
% \usepackage[margin=1in]{geometry}
\usepackage[utf8]{inputenc}

\usepackage{graphicx}
\graphicspath{ {images/} }

\usepackage[colorlinks = true,
            linkcolor = blue,
            urlcolor  = blue,
            citecolor = blue,
            anchorcolor = blue]{hyperref}

\begin{document}

\title{BrizMUN 2017 Briefing Paper \\ SPECPOL \\ Preventing Interstate Interference with Elections and Referendums}
\author{Maxwell Bo \& Dermot Aldenton}
\date{April 20, 2017}
\maketitle


\section{Background Information}

SPECPOL will investigate ways to prevent interstate interference in international elections and referendums. Corruption and fraud within domestic elections has been known to occur in developing countries, and they have been increasingly monitored by international bodies such as the United Nations. Interstate tampering and interference within other states’ domestic elections and referendums is an important issue especially within developing nations being influenced by larger more developed nations. SPECPOL will investigate current problems with election interference and the corruption and tampering of referendum results by interstate parties.

\subsection{The Committee}

The \textbf{Special Political and Decolonization Committee (SPECPOL)} deals with a variety of subjects which include those related to decolonisation, Palestinian refugees and human rights, peacekeeping, mine action, outer space, public information, atomic radiation and the University for Peace. 

General Assembly resolutions define important issues, reaffirm and condemn past actions, and recommend future action – in general, they do not have the jurisdiction to make binding decisions. Resolutions on this topic will require a simple majority (50\%+1) to pass.

\subsubsection{History}
\subsubsection{Scope}
\subsubsection{Methodology}

\section{The Issue}
\subsection{History of the Issue}

\subsection{Crux of the Issue}

\subsection{Rules of Debate}


\section{Further Research}

In particular, delegates should have a general understanding of their nation’s:

\begin{itemize}
	\item Existing policy on each subject area
	\item Political system and party in power
\end{itemize}

\end{document}
